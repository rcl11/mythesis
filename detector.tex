\chapter{The \CMS experiment}
\label{chap:detector}

%\chapterquote{There, sir! that is the perfection of vessels!}
%{Jules Verne, 1828--1905}

\section{The \LHC}
The Large Hadron Collider (\LHC) at \CERN is a new hadron collider,
located in the same tunnel as the Large Electron-Positron collider
(\LEP)~\cite{Brianti:2004qq}. Where \LEP's chief task was the use
of \unit{90--207}{\GeV} \epluseminus collisions to establish the
precision physics of electroweak unification\dots

\section{The \CMS experiment}
\label{sec:CMSInDetail}

The detector can be described in terms of the coordinate system conventionally used within CMS. The
origin is placed at the interaction point with the $z$ axis collinear with the
beam. Then the $x$ axis is chosen to point towards the centre of the LHC ring
and forms a plane with $y$, the remaining transverse coordinate perpendicular to
$x$ and $z$. The angle $\phi$ is the azimuthal angle with respect to the $x$
axis and $\theta$ is the polar angle in the $x-y$ plane. Another coordinate
often used is pseudorapidity, defined as $\eta = - \ln[\tan(\theta/2)]$. 


\section{Trigger system}
\label{sec:triggers}
