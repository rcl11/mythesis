\chapter{Hhh analysis}
\label{chap:Hhh}

As described in detail in the previous chapters, there are many popular MSSM
models which incorporate the 125 GeV Higgs. As seen in Chapter
\ref{chap:httmssm}, the $H\rightarrow\tau\tau$ analysis is very successful in 
setting limits on various MSSM models. The $H\rightarrow\tau\tau$ result primarily sets
limits on the high $\tan\beta$ regions, and so with large amounts of the
$m_{A}-\tan\beta$plane ruled out for such MSSM scenarios, focus shifts more 
to the regions which are still allowed, in particular the low $\tan\beta$.

In certain low $\tan\beta$ regions of the MSSM, the branching ratio for the
decay of the heavy neutral scalar Higgs, H, into two of the light Higgs, h,
BR($H\rightarrow hh$), is enhanced. Thus we could consider models in which the
light Higgs has a mass of 125 GeV and is the Higgs particle discovered at the
LHC, and as such we could get production of a pair of these light Higgs from the
heavier Higgs. The range of heavy Higgs masses in consideration is from 260 GeV
(driven by the kinematic threshold for the producton of two 125 GeV Higgs
bosons) up to 350 GeV (above which the branching ratio for Higgs decaying into
tops becomes overwhelmingly high).

When looking for two 125 GeV Higgs bosons, the final state consisting of two
$\tau$ leptons and two b quarks has some sensitivity. Hence we can use the
inclusive selection from the $H\rightarrow\tau\tau$ and require additional jets
to form the $h\rightarrow bb$. In this way we can use a large amount of the
expertise and methods from the $H\rightarrow\tau\tau$ analysis.


