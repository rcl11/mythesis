%% Title
\titlepage[Imperial College London\\Department of Physics]%
{A thesis submitted to Imperial College London\\
for the degree of Doctor of Philosophy}

\begin{abstract}[]
The copyright of this thesis rests with the author and is made available under a
Creative Commons Attribution Non-Commercial No Derivatives licence. Researchers
are free to copy, distribute or transmit the thesis on the condition that they
attribute it, that they do not use it for commercial purposes and that they do
not alter, transform or build upon it. For any reuse or redistribution,
researchers must make clear to others the licence terms of this work.
\end{abstract}

%% Abstract
\begin{abstract}%[\smaller \thetitle\\ \vspace*{1cm} \smaller {\theauthor}]
  %\thispagestyle{empty}
  The Compact Muon Solenoid (CMS) detector is located at the CERN Large Hadron
  Collider. It is a general purpose detector designed to test predictions of the
  Standard Model (SM) and search for the Higgs Boson, as well as search for
  signatures of new physics beyond the SM. The analyses described in
  this thesis use proton-proton collision data recorded by CMS during 2011 and
  2012. Searches for neutral Higgs bosons decaying into tau pairs are presented.
  The searches are performed in the context of the SM Higgs Boson or the neutral
  Higgs bosons of the minimal supersymmetric standard model (MSSM). The MSSM
  searches are separated into two analysis. The first contains a direct search 
  for any one of the MSSM neutral Higgs bosons in the mass range from $90\,\GeV$ to 
  $1\,\TeV$ decaying directly into taus. The second searches for the signature 
  from a heavy neutral Higgs boson of the MSSM decaying into two light MSSM Higgs 
  bosons with SM-like properties in the final state with two b-quarks and two
  taus. The SM and MSSM $\Pgt\Pgt$ analyses use $4.9\,\invfb$ of data collected
  at $7\,\TeV$ and $19.7\,\invfb$ collected at $8\,\TeV$. In the SM analysis, an
  excess of events in data above the background expectation is observed with an
  observed (expected) local significance of $3.0$ ($3.1$) standard deviations at
  $125\,\GeV$. This excess is found to be consistent with the $125\,\GeV$ Higgs
  boson of the SM in mass and coupling strength. No significance excess is
  observed in either MSSM search. For both MSSM analyses, upper limits
  are set on cross-section times branching ratio for the Higgs production
  processes at the $95\%$ confidence level. For the MSSM $\Pgt\Pgt$ analysis, 
  limits are also set in $m_{\PA}$-$\tan\beta$ parameter space of several interesting
  scenarios.
\end{abstract}


%% Declaration
\begin{declaration}
  I declare that the work contained in this thesis is my own. Where figures and
  results are taken from other sources this is indicated by an appropriate
  reference. All figures labelled ``CMS'' are sourced directly from CMS
  publications, including those made by myself, and as such include the
  relevent reference in the caption. The label ``CMS (unpublished)'' or ``CMS
  Preliminary'' is reserved for figures not directly included in a paper but 
  made public via a public twiki page or other preliminary public document.
  Studies conducted and results produced by myself are indicated in the main
  body of the text. For the analyses in chapters \ref{chap:htt-sm} to \ref{chap:Hhh}
  the work was conducted as part of the CMS $\PH\to\Pgt\Pgt$ working group. I
  contributed to measurement of identification and trigger efficiencies and
  the statistical interpretation for the results in chapters \ref{chap:htt-sm}
  and \ref{chap:htt-mssm}. These results are those from the most recent legacy results
  from Run 1 of the LHC~\cite{HIG-13-004,HIG-13-021}, and my work
  was also included in the preliminary results that predated them with subsets
  of the Run 1 data available at the time; at the HCP 2012
  \cite{CMS-PAS-HIG-12-050,CMS-PAS-HIG-12-043}
  and Moriond 2013~\cite{CMS-PAS-HIG-13-004-mor} conferences, as well as for a preliminary result
  produced in November 2013 \cite{CMS-PAS-HIG-13-021}. The final analysis in
  chapter \ref{chap:Hhh} is to date not yet published. For this
  analysis, my work included designing event selections, studying
  background methods and evaluating systematics and further work on statistical
  interpretation. For this analysis I provided the main result for two out of
  the three channels included in the final version of the analysis to be published.
  \vspace*{1cm}
  \begin{flushright}
    Rebecca Lane
  \end{flushright}
\end{declaration}


%% Acknowledgements
\begin{acknowledgements}
  I would like to thank the STFC and Imperial College for giving me the
  opportunity to conduct this research, and especially for providing me with two
  years of valuable experience working at CERN. Thanks to my supervisor David
  Colling for persuading me towards the exciting field of Higgs physics and for
  guiding me through my PhD research. I also thank M\'{o}nica for her
  support and advice during my time at CERN, and Sasha for his friendly
  enthusiasm. I owe a huge amount to Andrew, for teaching me the important
  aspects of an analysis with exceptional patience. I thank all of the friends I
  made on LTA at CERN for providing some extremely fun distractions from work, and those
  back home for not minding when I missed all their social events while I was
  away. Finally I thank my parents, Kevin and Shirley, my brother Tom and my
  boyfriend Tom for their love and support.  
\end{acknowledgements}


%% Preface
%\begin{preface}
%  This thesis describes my research on various aspects of the \LHCb
%  particle physics program, centred around the \LHCb detector and \LHC
%  accelerator at \CERN in Geneva.

% \noindent
%  For this example, I'll just mention \ChapterRef{chap:SomeStuff}
%  and \ChapterRef{chap:MoreStuff}.
%\end{preface}

%% ToC
\tableofcontents
\listoffigures
\listoftables

%% Strictly optional!
%\frontquote%
%{Writing in English is the most ingenious torture\\
%   ever devised for sins committed in previous lives.}%
%  {James Joyce}
