\chapter{Search for \ac{MSSM} $\Pphi\to\Pgt\Pgt$}
\label{chap:httmssm}

As discussed in chapter \ref{chap:theory}, from the \ac{LHC} we have observation
of a $125~\GeV$ Higgs boson which is so far consistent with the \ac{SM}. From
chapter \ref{chap:htt-sm}, we have evidence at the $3\sigma$ level of a Higgs
boson decaying into tau leptons, which has properties consistent with the $125~\GeV$ boson
discovered in the other decay channels and consistent with the \ac{SM}.
However, as discussed in section \ref{sec:mssmhiggs}, the \ac{MSSM} is an
alternative theory to the \ac{SM} which could provide a $125~\GeV$ Higgs boson
consistent with the properties observed so far experimentally. In this theory,
we would also see two additional neutral Higgs bosons. This chapter describes
the search for the three neutral Higgs bosons of the \ac{MSSM} in the final
state of $\Pgt\Pgt$. We use the symbol $\Pphi$ to refer to any one of the three
neutral Higgs bosons, $\PH$, $\Ph$ or $\PA$. This analysis largely uses the same
techniques as the \ac{SM} $\HToTauTau$ analysis and as such this chapter focusses on the
places that the analysis is different and on the interpretation of results in
the context of the \ac{MSSM}. The results documented are the ``legacy''
\ac{MSSM} $\Pphi\to\Pgt\Pgt$ result from run 1 of the \ac{LHC} \cite{HIG-13-021}, and like for
chapter~\ref{chap:htt-sm} the information is focussed on the analysis in the
$\etau$ and $\mutau$ final states. The results shown in
section~\ref{sec:mssmresults} include the combination of all channels, which for
the \ac{MSSM} analysis includes the $\etau$, $\mutau$, $\emu$, $\tautau$ and
$\mumu$ final states. 

\section{Event Selection and Categorisation}
\label{sec:mssmEventSelection}

The inclusive selection of the candidate di-tau pair is almost exactly the same
as that used for the \ac{SM} $\HToTauTau$ analysis described in section
\ref{sec:eventSelection}, with one small exception. Due
to the fact that the $\Pgth$ $\pt$ is not directly used in the \ac{MSSM} analysis we are
able to lower the $\Pgth$ $\pt$ threshold from $30~\GeV$ to $20~\GeV$. This is useful
in the \ac{MSSM} analysis since we are interested in Higgs bosons of a
much larger mass up to $1~\TeV$ and so we must consider events up to a large
$m_{\Pgt\Pgt}$. A lower $\pt$ cut on the $\Pgth$ gives a larger overall number of events
and improves the statistics in the background templates across the $m_{\Pgt\Pgt}$ range. 
Using hadronic taus lower than $30~\GeV$ in the \ac{SM} analysis was not possible 
due to an observed data - \ac{MC} discrepancy in the $\pt$ distribution for low 
$\pt$ taus as a result of imperfect modelling of the trigger in \ac{MC}.

An alternative event categorisation is used for the \ac{MSSM} analysis. Much
like in the \ac{SM} analysis where the categorisation is used to target the different 
production modes of the Higgs, the \ac{MSSM} analysis follows a similar
strategy. Events are split into those with at least one b-tagged jet, or exactly
0 b-tagged jets, referred to as the b--tag and no b--tag category.  The
definition for a b-tagged jet is as described in section~\ref{sec:btag} and has
$\pt > 30~\GeV$ and $|\eta| < 2.4$. The b--tag
category is more sensitive to b-associated Higgs production, and the no b--tag
to gluon fusion. Figure \ref{fig:nbtag} shows the number of b-tagged jets in the
$\mutau$ channel, with the two signal contributions overlaid. It can be seen
that a large amount of the b-associated production signal still falls into the
no b--tag category, which occurs for events where the b-jets fall outside the
acceptance.

\begin{figure}[tbh]
\includegraphics[width=0.6\textwidth]{plots/htt-mssm/n_bjets_inclusive_mt_2012_log.pdf}

\caption{Number of b-tagged jets in the $\mutau$ channel, as used to separate
events into b--tag and no b--tag categories. Signal contributions are shown
separately for b-associated producton and gluon fusion.}
\label{fig:nbtag}
\end{figure}

\section{Datasets and \ac{MC} samples}
\label{sec:mssmdataandMC}

The datasets and \ac{MC} samples for each of the background processes are
identical to those described in section \ref{sec:dataandMC}. The signal is
generated using \textsc{pythia}~\cite{Sjostrand:2006za} at \ac{LO}. Like the
samples described in section \ref{sec:dataandMC}, it uses \textsc{pythia}
for parton showering and hadronisation and \textsc{tauola}~\cite{TAUOLA} for tau
decays. Additional proton-proton collisions to simulate pileup events are added
as for the other samples. The signal samples are generated in a mass range from
$90~\GeV$ up to $1~\TeV$ in steps of varying size. 

\section{Background Methods and Systematics}
\label{sec:mssmBackgroundsSysts}

\subsection{Background Methods}
\label{sec:mssmBackgrounds}
The background composition is very similar to that of the \ac{SM} $\HToTauTau$
analysis and the methods used to estimate the contributions follow those
described in section \ref{sec:backgrounds}. The requirement of at least one
b-tagged jet in the b-tag category reduces background from $\ZToTauTau$
and increases the contribution of $\ttbar$. As described in section
\ref{sec:backgroundEstimation_Ztautau}, an embedding procedure is used for the
$\ZToTauTau$ estimate, in which $\PZ\to\Pgm\Pgm$ events in data are replaced
with simulated taus. It is known that a small fraction of selected data events
are $\ttbar$ events instead of $\PZ\to\Pgm\Pgm$ events, and hence it is
necessary to calculate this contamination in the b--tag category so as to avoid
double counting of $\ttbar$. The contamination is estimated by running the
embedding procedure on a $\ttbar$ \ac{MC} sample. For the $\etau$ and $\mutau$ channels, this
contamination is around $1.5\%$, and so the $\ttbar$ yield is reduced
accordingly. 

Similarly to the way cuts on $m_{jj}$ and $|\Delta\eta_{jj}|$ are relaxed to
obtain smooth shapes for the $\WJets$ background in the VBF categories, the
b-tagging working point for the jets is relaxed to the loose working point to
obtain the shape for the b--tag category. This is also done for the shape of
the $\PZ\to\ell\ell$ background. For the QCD, the same-sign data is used for
both shape and normalisation in the no--btag category, whereas for the b--tag
category the shape is taken from anti-isolated same-sign data using the relaxed
b-tagging working point.

Data to \ac{MC} corrections are the same as those used in the \ac{SM}
$\HToTauTau$ analysis as described in section~\ref{sec:datamcfactors}. An
additional correction is derived for the $\WJets$ background to account for
observed differences in the jet-tau fake-rate at high $\pt$, which in particular
affects the high mass tail of the $\WJets$. 

\subsection{Tail fitting of backgrounds}
\label{sec:tailfitting}

One large difference between the \ac{MSSM} $\Pphi\to\Pgt\Pgt$ analysis compared
with the $\HToTauTau$ analysis is the fact that we consider Higgs bosons of
masses up to $1~\TeV$. This means that we must study the $m_{\Pgt\Pgt}$
distribution up to high values of around $1.5~\TeV$. In these high mass regions, the
number of events in our backgrounds is greatly reduced, and it becomes more
difficult to obtain smooth background templates. This results in a need for a
large number of bin-by-bin uncertainties like those described in
section~\ref{sec:systematicUncertainties_shape} to cover bins with low
statistics. A better method for dealing with this is to ensure that the bins are
all populated by fitting the template in the high mass region using an analytic
function, and replacing the template with the result of that function.

The function used for the high mass fits takes the following form:

\begin{equation}
f = exp\left(\frac{-m_{\Pgt\Pgt}}{c_{0} + c_{1}\cdot m_{\Pgt\Pgt}}\right) ,
\end{equation}

where $c_{0}$ and $c_{1}$ are free parameters in the fit. The fit is made to
the \ac{MC} in the region $m_{\Pgt\Pgt} > 150~\GeV$ and then the template is
replaced by the values of the analytical function for this region. To represent
the uncertainty on this high mass fit in the final maximum-likelihood fit, shape
uncertainties are generated corresponding to the $\pm1\sigma$ shift in the fit
parameters $c_{0}$ and $c_{1}$. The values of the uncertainty $\sigma$ are the 
eigenvalues of the covariance matrix of the fit. Figure \ref{fig:tailfits} shows
an example of the fit obtained for the \WJets background in the $\mutau$
channel, showing the central fit to the template and the shape nuisances which
are added to the maximum likelihood fit.

\begin{figure}[tbh]
\subfloat[]{
\includegraphics[width=0.5\textwidth]{plots/htt-mssm/W_fine_binning_CMS_shift1_muTau_nobtag_8TeV_Rebin.pdf}}
\subfloat[]{
\includegraphics[width=0.5\textwidth]{plots/htt-mssm/W_fine_binning_CMS_shift1_muTau_btag_8TeV_Rebin.pdf}}
\caption{Analytic fits to the high mass tail of the $m_{\Pgt\Pgt}$ distribution,
shown here for the example of the $\WJets$ background. Fits are shown for the
no--btag (a) and b--tag (b) categories of the $\mutau$ channel in $8~\TeV$
\ac{MC}. The green line corresponds to the central fit, and the red and blue
lines indicate the systematic uncertainties added to the final
maximum-likelihood fit corresponding to the uncertainties in the two fitted
parameters.}
\label{fig:tailfits}
\end{figure}

Fits are performed for the $\WJets$, QCD and $\ttbar$ backgrounds in the $\etau$
and $\mutau$ channels, where the statistics in the templates are good enough to
produce a reliable fit. In the backgrounds in which a tail fit is not used, and
in the low mass regions of the fitted backgrounds, bin-by-bin uncertainties as
described in section~\ref{sec:systematicUncertainties_shape} are applied.

\subsection{Other Systematic Uncertainties}
%%Try to find some references for some of these things
The majority of the systematic uncertainties in the \ac{MSSM} analysis are the
same as those in the \ac{SM} analysis as described in
section~\ref{sec:systematics}, with evaluation in the b--tag and no
b--tag categories where appropriate. An uncertainty equal to the magnitude of
the correction to the $\ttbar$ as a result of the embedding contamination
is taken as an uncertainty in the rate of $\ttbar$ events. A shape 
uncertainty for the correction for jet-tau fake-rate on the $\WJets$ background
is also applied, where the shapes are generated by shifting the fake-rate
correction up and down by the uncertainty in its measurement. 

Another shape uncertainty is included to account for differences in tau ID
efficiency at high $\pt$, affecting the high $m_{\Pgt\Pgt}$ events. 
Uncertainties on the signal are similar to those on the \ac{SM} signal, and vary
with $m_{\PA}$ and $\tan\beta$. The \ac{PDF} uncertainties range from $2$--$10\%$ and scale uncertainties range from
$5$--$25\%$ for gluon-gluon fusion and $8$--$15\%$ for b-associated production
\cite{CMS-PAS-HIG-13-021}.

\section{Results}
\label{sec:mssmResults}

\subsection{Signal Extraction}
\label{sec:mssmSignalExtraction}

The discriminating variable used for signal extraction is the same as in the
\ac{SM} analysis - the di-tau mass. Other than the fact that the distribution is
included in the fit up to $1.5~\GeV$, the maximum likelihood fit is the same as
described in section~\ref{sec:} using equations~\ref{eq:LikelihoodFunction} and
\ref{eq:PoissonDistribution}.

Figure \ref{fig:mssmpostfitmass} shows the di-tau mass distribution in the
$\etau$ and $\mutau$ channels for the b-tag and no-btag categories. The plots
are shown on a logarithmic scale to highlight the tail of the distribution of
interest in the \ac{MSSM} analysis. 

\begin{figure}[tbh]
\subfloat[]{
\includegraphics[width=0.5\textwidth]{plots/htt-mssm/muTau_nobtag_postfit_7TeV_8TeV_LOG.pdf}}
\subfloat[]{
\includegraphics[width=0.5\textwidth]{plots/htt-mssm/muTau_btag_postfit_7TeV_8TeV_LOG.pdf}}

\subfloat[]{
\includegraphics[width=0.5\textwidth]{plots/htt-mssm/eleTau_nobtag_postfit_7TeV_8TeV_LOG.pdf}}
\subfloat[]{
\includegraphics[width=0.5\textwidth]{plots/htt-mssm/eleTau_btag_postfit_7TeV_8TeV_LOG.pdf}}
\caption{Post-fit $m_{\Pgt\Pgt}$ distributions for the no-btag
(left) and b-tag categories. Plots are shown for
the $\mutau$ channel (top) and $\etau$ channel (bottom), for the combination of
7 and $8~\TeV$ data \cite{HIG-13-021}.}
\label{fig:mssmpostfitmass}
\end{figure}

The \ac{MSSM} analysis differs from the \ac{SM} analysis when we 
interpret the result of the maximum likelihood fit in the form of a limit. This
is done differently for the two types of limit producted - referred to as `model 
independent' and `model dependent'. 

\subsection{Model Independent limits}
\label{sec:modelindependent}

The simplest type of limit produced in the \ac{MSSM} analysis is analogous to the 
expected limit on $\mu$ produced in the \ac{SM} analysis (figure
\ref{fig:results-limit}). In the \ac{MSSM} analysis, instead of setting a limit
on the signal strength modifier with reference to a benchmark cross-section, a
more 'model-independent' limit is given on cross-section times branching ratio
for the Higgs production process. This can then be interpreted in many different
benchmark models. A limit is set separately for each of the two dominant production modes,
b-associated production and gluon fusion production. It is not possible to
completely disentangle the two production modes - as shown in
section~\ref{sec:mssmEventSelection} the categorisation does not completely
separate the signal contributions, and some b-associated production signal is
found in no b--tag category and some gluon fusion in the b--tag category. Hence
to produce a limit on each process separately, the other signal process is
`profiled'. This means that it is allowed to float in the fit like the nuisance
parameters. This type of limit is referred to as a `single-resonance' search,
due to the fact that there is no requirement that the three Higgs bosons of the
\ac{MSSM} are included, the search is simply for the production of $\Pphi$,
which represents any of the three bosons.

Figure \ref{fig:mssmModelIndependent} shows the expected and observed limit on
cross-section times branching ratio for the gluon fusion and b-associated
production processes. The expected limit is generated in the same way as the
expected limit in figure \ref{results-limit} b), by injecting an \ac{SM} Higgs.
In this way the observed limit is compared to an expected including both the
backgrounds and an \ac{SM} Higgs. 

\begin{figure}[tbh]
\subfloat[]{
\includegraphics[width=0.5\textwidth]{plots/htt-mssm/cmb_ggH-limit.pdf}}
\subfloat[]{
\includegraphics[width=0.5\textwidth]{plots/htt-mssm/cmb_bbH-limit.pdf}}
\caption{Limits on cross-section times branching ratio for a) gluon fusion Higgs
production and b) b-associated Higgs production for the combination of all
channels and categories. For each limit the other production process is
profiled. The observed limit is compared with an expected limit which includes
the \ac{SM} Higgs.}
\label{fig:mssmModelIndependent}
\end{figure}

The reason for comparing the observed with an expected including the \ac{SM} Higgs 
is that we cannot interpret results in the context of the \ac{MSSM} without taking into account
the fact that we have $3\sigma$ evidence of an \ac{SM}-like Higgs boson decaying
into taus. Despite the fact that the categorisation of events is chosen to
enhance selection of an \ac{MSSM} signal and not an \ac{SM} signal, the
selection is still very close to that of the \ac{SM} analysis and hence there is
still some sensitivity to the \ac{SM} Higgs in the \ac{MSSM} analysis. Thus we
have to interpret any excess in data over background very carefully, since it
does not automatically mean evidence of an \ac{MSSM} signal when it could be
from the \ac{SM} Higgs. 

\subsection{Model Dependent limits}
\label{sec:modeldependent}

The sensitivity to the \ac{SM} Higgs becomes even more important when
considering a model-dependent result. As discussed in
section~\ref{sec:MSSMBenchmarks}, the \ac{MSSM} benchmark scenarios being tested in this
analysis are required to be consistent with experimental measurements of the
$125~\GeV$ Higgs bosons. Hence in such scenarios, one of the three neutral Higgs
bosons must be very similar to the \ac{SM} $125~\GeV$ Higgs boson. This means
that by construction an \ac{MSSM} signal from one of these benchmarks looks a
lot like the \ac{SM} signal, with the exception of the requirement of the
additional two Higgs bosons.

Hence when interpreting the results of the \ac{MSSM} search in the context of an
\ac{MSSM} benchmark scenario, we cannot simply construct a limit from
considering the background-only hypothesis compared with the
signal-plus-background hypothesis. Instead, we must build a limit based on
whether the data agrees better than the background plus \ac{SM} signal or
background plus \ac{MSSM} signal hypothesis.

\subsubsection{\ac{MSSM} and \ac{SM} hypothesis testing}

Insert statistics here


\begin{figure}[tbh]
\includegraphics[width=0.7\textwidth]{plots/htt-mssm/sigsep_13.pdf}
\caption{Distributions of the test statistic for toys with \ac{MSSM} or \ac{SM}
signal for $m_{\PA} = 300~\GeV$, $\tan\beta = 13$ in the $m_{h}^{\text{max}}$
scenario. The black vertical line indictates the observed value of the test
statistic. The separation of the distributions is such that this point can be
excluded at 95$\%$ C.L.}
\label{fig:toydistribution}
\end{figure}


Figure \ref{fig:hypotestcompare} shows the

\begin{figure}[tbh]
\subfloat[]{
\includegraphics[width=0.5\textwidth]{plots/htt-mssm/cmb_mhmax-mA-tanb-SMinjected.pdf}}
\subfloat[]{
\includegraphics[width=0.5\textwidth]{plots/htt-mssm/cmbRL_mhmax-HypoTest.pdf}}
\caption{Expected and observed limit in the $m_{\PA}-\tan\beta$ plane of the
$m_H^{\text{max}}$ scenario. In the left hand plot, the \ac{MSSM} signal is
compared with the background only hypothesis. In the right hand plot, hypothesis
separation testing compares the \ac{MSSM} hypothesis with the \ac{SM}
hypothesis \cite{,HIG-13-021}.}
\label{fig:hypotestcompare}
\end{figure}

\begin{figure}[tbh]
\subfloat[]{
\includegraphics[width=0.5\textwidth]{plots/htt-mssm/cmbRL_mhmodp-HypoTest.pdf}}
\subfloat[]{
\includegraphics[width=0.5\textwidth]{plots/htt-mssm/cmbRL_mhmodm-HypoTest.pdf}}
\caption{Expected and observed limit in the $m_{\PA}-\tan\beta$ plane of the
$m_H^{\text{mod+}}$ scenario (a) and $m_H^{\text{mod-}}$ scenario (b). Hypothesis
separation testing is used to compare the \ac{MSSM} hypothesis with the \ac{SM}
hypothesis. The red area indicates the region of phase space which already
excluded by the Higgs mass constraint of $125\pm3~\GeV$ \cite{HIG-13-021}.}
\label{fig:mhmodpmhmodm}
\end{figure}

\begin{figure}[tbh]
\subfloat[]{
\includegraphics[width=0.5\textwidth]{plots/htt-mssm/cmbRL_lightstau1-HypoTest.pdf}}
\subfloat[]{
\includegraphics[width=0.5\textwidth]{plots/htt-mssm/cmbRL_lightstopmod-HypoTest.pdf}}
\caption{Expected and observed limit in the $m_{\PA}-\tan\beta$ plane of the
light-stau scenario (a) and light-stop scenario (b). Hypothesis
separation testing is used to compare the \ac{MSSM} hypothesis with the \ac{SM}
hypothesis. The red area indicates the region of phase space which already
excluded by the Higgs mass constraint of $125\pm3~\GeV$ \cite{HIG-13-021}.}
\label{fig:lightstaulightstop}
\end{figure}

\begin{figure}[tbh]
\subfloat[]{
\includegraphics[width=0.5\textwidth]{plots/htt-mssm/cmbRL_tauphobic-HypoTest.pdf}}
\subfloat[]{
\includegraphics[width=0.5\textwidth]{plots/htt-mssm/cmbRL_lowmH-HypoTest.pdf}}
\caption{Expected and observed limit in the $m_{\PA}-\tan\beta$ plane of the
$\tau$-phobic scenario (a) and low-$m_{\PH}$ scenario (b). Hypothesis
separation testing is used to compare the \ac{MSSM} hypothesis with the \ac{SM}
hypothesis. The red area indicates the region of phase space which already
excluded by the Higgs mass constraint of $125\pm3~\GeV$ \cite{HIG-13-021}.}
\label{fig:tauphobiclowmH}
\end{figure}
