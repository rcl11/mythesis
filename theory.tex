\chapter{Theory}
\label{chap:theory}

\section{The Standard Model of Particle Physics}
\label{sec:theSM}

The \ac{SM} of fundamental interactions is the theory
which describes the electromagnetic, weak and strong nuclear interactions, often
represented as a collection of the elementary particles predicted by the theory.
The SM has successfully described data from a wide range of experiments, and in the 
years prior to the LHC, all of the predicted elementary particles were verified 
except for one: the Higgs boson.

The \ac{SM} is a renormalisable quantum field theory, in which the constituents of
matter are represented as spin-$\frac{1}{2}$ fermions which interact with
forces mediated by spin-1 bosons. These interactions are described by a
Lagrangian which is invariant under $SU(3)_{C} \times SU(2)_{L} \times U(1)_{Y}$
symmetries. The $SU(3)_{C}$ part describes the strong interaction, mediated by
particles which carry colour charge (C). In these interactions, described by the
theory of \ac{QCD}\cite{}, the force mediators are massless gluons
and the coloured fermions are the quarks. The remaining fundamental fermions,
the leptons, do not carry colour charge and hence do not interact via the strong
force. Both the leptons and quarks participate in electroweak interactions,
which are governed by the $SU(2)_{L} \times U(1)_{Y}$ symmetry.

The electroweak symmetry describes the unified electromagnetic and weak
interactions\cite{}, and was one of the major achievements of the twentieth
century in the \ac{SM}. In electroweak theory the quantum numbers of interest are
weak hypercharge $y$ and weak isospin $t_{1,2,3}$. These are related to the
electric charge Q as: 

\begin{equation}
Q = t_{3} + \frac{y}{2}
\end{equation}

The gauge fields associated with these fields are the three weak isospin fields,
$W_{\mu}^{i}$, $i = 1,2,3$, and the hypercharge field $B_{\mu}$.


\cite{GlashowPartialSymmetries,WeinbergModelOfLeptons,SalamNobelSymposium}.

\subsection{The Higgs Mechanism in the SM}
\label{sec:SMHiggs}

\section{Theories beyond the SM}
\label{sec:BSM}

Despite its successes, the \ac{SM} is known to have some shortcomings. One concerns
the calculations of the Higgs Boson mass, and is known as the Hierarchy problem. 

\subsection{The Higgs sector in the MSSM}

\subsection{MSSM Models incorporating the LHC Higgs}

\subsection{Other possibilities}



