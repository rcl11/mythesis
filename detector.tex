\chapter{The \CMS experiment}
\label{chap:detector}

\section{The \LHC}
\label{sec:theLHC}

The Large Hadron Collider (LHC) \cite{Evans:2008zzb} is synchrotron acceralator
with a $27 km$ circumference designed to
collide beams of protons at centre of mass energies as high as $14~\TeV$. It is
hosted in the former tunnel of the Large Electron Positron (LEP) \cite{LEP:1983aa} experiment on the French-Swiss border
near Geneva and operated by the European Organisation for Nuclear Research
(CERN). As well as proton-proton collisions, the LHC also accelerates beams of
lead ions to produce both lead-lead (PbPb) and proton-lead (pPb) collisions.

The protons originate from hydrogen gas, the atoms of which are stripped of
their electrons using an electric field. The protons are then accelerated to an
energy of $50~\MeV$ in the Linac 2 accelerator. Proton bunches are formed inside
the \ac{PSB}, increasing the energy to $1.4~\GeV$. The \ac{PS} creates proton
beams from the bunches, increasing the energy to $26~\GeV$. Further acceleration
in the \ac{SPS} raises the beam energy to $450~\GeV$ before being injected into
the LHC. The LHC contains two beams circulating in opposite directions. The
design operation conditions consist of each beam containing up to 2800 bunches
spaced $25~\ns$ apart and made up of $\mathcal{O}(10^{11})$ protons each. 
Approximately 1200 superconducting dipole magnets keep the beams circulating
before collisions. The beams collide at four points around the LHC where they
are recorded by four detectors: ALICE \cite{Aamodt:2008zz}, ATLAS
\cite{Aad:2008zzm}, CMS \cite{Chatrchyan:2008aa} and LHCb \cite{Alves:2008zz}


\section{The \CMS experiment}
\label{sec:CMSInDetail}

The detector can be described in terms of the coordinate system conventionally used within CMS. The
origin is placed at the interaction point with the $z$ axis collinear with the
beam. Then the $x$ axis is chosen to point towards the centre of the LHC ring
and forms a plane with $y$, the remaining transverse coordinate perpendicular to
$x$ and $z$. The angle $\phi$ is the azimuthal angle with respect to the $x$
axis and $\theta$ is the polar angle in the $x-y$ plane. Another coordinate
often used is pseudorapidity, defined as $\eta = - \ln[\tan(\theta/2)]$. 


\section{Trigger system}
\label{sec:triggers}
