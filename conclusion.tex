\chapter{Conclusions}
\label{chap:conclusion}

Three analyses have been presented in this thesis using proton-proton collision
data recorded by the CMS detector during Run 1 of the LHC. The analyses include
searches for a neutral Higgs boson in the context of the \ac{SM} and \ac{MSSM}
in the $\Pgt\Pgt$ final state. This channel provides direct access to the Yukawa
couplings between fermions and the Higgs field. The \ac{SM} and \ac{MSSM}
$\PH\to\Pgt\Pgt$ analyses share a common baseline and many analysis techniques,
both making use of $m_{\Pgt\Pgt}$ as discriminating variable for signal
extraction. The \ac{MSSM} $\PH\to\Ph\Ph$ analysis uses the same baseline with
the exception of requiring at least two jets to form the $\Ph\to\Pqb\Pqb$ part.
For maximal sensitivity, cuts on $m_{\Pgt\Pgt}$ and $m_{\Pqb\Pqb}$ are used in a
window around $125\,\GeV$ before using the 4 body mass of the two taus and the 
two b-quarks evaluated using a kinematic fit as the variable for
signal extraction. All three analyses make use of event categorisation to target 
the signatures of the Higgs production process and decay and separate more signal-like and
background-like events.

An excess of data over the background prediction is seen in the \ac{SM}
analysis, with local observed (expected) significance of (3.2$\sigma$)
3.7$\sigma$. This excess is found to be consistent with predictions for an
\ac{SM} Higgs boson with a mass of $125\,\GeV$, both in measurements of the
couplings and the mass. No significant excess of events is seen in either of the
\ac{MSSM} analyses. This results in upper limits at $95\%$ CL on the
cross-section times branching ratio for \ac{MSSM} Higgs production processes. In
the \ac{MSSM} $\HToTauTau$ analyses limits are also set in interesting benchmark
scenarios which incorporate a $125\,\GeV$ Higgs boson. For the interpretation of
the \ac{MSSM} results, the possible contribution from an \ac{SM} Higgs in the
dataset is taken into account. 

The discovery of a Higgs boson with a mass around $125\,\GeV$ was a huge success 
for the LHC and the theory of electroweak symmetry breaking. Establishing
whether or not this particle is the Higgs boson of the \ac{SM} or from some
other theory remains an extremely interesting field. Later this year, the LHC
will re-commence collisions at a centre of mass energy of $13\,\TeV$. Data from
Run 2 will be used to continue to test the compatibility of the Higgs boson with
the \ac{SM}.

%The discovery of a Higgs boson with a mass around $125\,\GeV$ was a huge success 
%for the LHC and the theory of electroweak symmetry breaking. Establishing
%whether or not this particle is the Higgs boson of the \ac{SM} or from some
%other theory remains an extremely interesting field. Later this year, the LHC
%will re-commence collisions at a centre of mass energy of $13\,\TeV$. Data from
%Run 2 will be used to continue to test the compatibility of the Higgs boson with
%the \ac{SM}.
